% In this file you should put the actual content of the blueprint.
% It will be used both by the web and the print version.
% It should *not* include the \begin{document}
%
% If you want to split the blueprint content into several files then
% the current file can be a simple sequence of \input. Otherwise It
% can start with a \section or \chapter for instance.

\begin{abstract}
  We outline a plan of formalization for partial combinatory algebras (PCA), including combinatory completeness,
  programming with PCAs, and some examples of PCAs. Time permitting, we will formalize the typed version as well.
\end{abstract}


This blueprint, and more generally the \href{https://github.com/andrejbauer/partial-combinatory-algebras}{\texttt{partial-combinatory-algebras}} Lean project, is part of the teaching materials for the course \href{https://www.andrej.com/zapiski/MAT-FORMATH-2024/book/}{Formalized mathematics and proof assistants}.
It serves as a model blueprint to help students create their own class project blueprints.

\section{Basic definitions}

The set of \emph{partial elements} $\Part{A}$ in $A$ is
%
\begin{equation*}
  \Part{A} = \set{ u \subseteq A \such \all{x, y \in y} x = y}.
\end{equation*}
%
We think of $\emptyset$ as the undefined value and $\set{a}$ the defined, or \emph{total} value $a \in A$.
We construe each $a \in A$ also as the element $\set{a} \in \Part{A}$, i.e., there is an implicit coercion $A \to \Part{A}$.

Given $u \in \Part{A}$, write $\defined{u}$ to mean that $u$ is defined, i.e., $\some{a \in A} a \in u$.

Initial experimentation with formalization shows that it is advantageous to work with partial elements rather than elements, as that avoids coercions and helps the rpoof assistant compute types. When a partial element should be total this is expressed as a separate claim.

\begin{definition}
  \label{def:partial-application}
  %
  \lean{PartialApplication}
  \leanok
  %
  A \emph{partial application} on a set $A$ is a map ${-} \cdot_A {-} : \Part{A} \times \Part{A} \to \Part{A}$.
\end{definition}

We also write $u \, v$ and $u \cdot v$ in place of $u \cdot_A v$ and we associate application to the left, so that $u \cdot v \cdot w = (u \cdot v) \cdot w$.

The preceding defintion is non-standard, as one usually expects ${-} \cdot_A {-} : A \times A \to \Part{A}$.
However, as explained above, we prefer our definition as it is easier to work with. You might also have expected
strictness conditions stating that $\defined{u \cdot_A v}$ implies $\defined{u}$ and $\defined{v}$, which we did have in the initial formalization, but they were never used. It looks like our formalization already lead to a small discovery, namely that partial combinatory algebras work just as well when its elements can be applied to undefined values.

\begin{definition}
  \label{def:PCA}
  %
  \leanok
  \lean{PCA}
  \uses{def:partial-application}
  %
  A set $A$ with a partial map ${-} \cdot_A {-}$ is a \emph{partial combinatory algebra (PCA)} if there are
  elements $\combK, \combS \in \Part{A}$ such that, for all $u, v, w \in \Part{A}$:
  %
  \begin{align*}
    &\defined{K}, &
    &\defined{S},\\
    &\defined{u} \lthen \defined{\combK \, u}, &
    &\defined{u} \lthen \defined{\combS \, u}, \\
    &\defined{u} \land \defined{v} \lthen \defined{\combK \, u \, v}, &
    &\defined{u} \land \defined{v} \lthen \defined{\combS \, u \, v}, \\
    &\defined{u} \land \defined{v} \lthen \combK \, u \, v = u, &
    &\defined{u} \land \defined{v} \land \defined{w} \lthen \defined{\combS \, u \, v \, w}, \\
    &&
    &\defined{u} \land \defined{v} \land \defined{w}
     \lthen \combS \, u \, v \, w = (u \, w) \, (v \, w), \\
  \end{align*}
\end{definition}


\section{Combinatory completeness}


\begin{definition}
  \label{def:expression}
  %
  \leanok
  \lean{Expr}
  %
  Given a set $\Gamma$ of \emph{variables} and a set $A$ of elements, the
  \emph{expressions} $\Expr \Gamma A$ are generated inductively by the following clauses:
  %
  \begin{itemize}
    \item the constants $\mathtt{K}$ and $\mathtt{S}$ are expressions,
    \item an element $a \in A$ is an expression,
    \item a variable $x \in \Gamma$ is an expression,
    \item if $e_1$ and $e_2$ are expressions then $e_1 \cdot e_2$ is an expression.
  \end{itemize}
\end{definition}

\noindent
The application $e_1 \cdot e_2$ in the third clause of the definition is formal, i.e., it is just a constructor of an inductive datatype. We again deviated from the standard definition in an inessential way by including primitive constants $\combK$ and $\combS$, separate from the corresponding elements of~$A$.

Given a set of variables $\Gamma$ and a set~$A$, a \emph{valuation} is a map $\eta : \Gamma \to A$ which assigns elements to variables.

\begin{definition}
  \label{def:override}%
  \leanok
  \lean{Expr.override}
  %
  Given a valuation $\eta : \Gamma \to A$, $x \in A$ and $a \in a$, we may \emph{override} $\eta$
  to get a valuation $\override \eta x a : \Gamma \to A$ such that
  %
  \begin{equation*}
    (\override \eta x a) y =
    \begin{cases}
      \eta(y) &\text{if $y \in \Gamma$,}\\
      a       &\text{if $y = x$.}
    \end{cases}
  \end{equation*}
\end{definition}

\begin{definition}
  \label{def:evaluation}
  %
  \leanok
  \lean{Expr.eval}
  %
  \uses{def:expression, def:PCA}
  %
  The \emph{evaluation} $\eval e \eta$ of an expression $e \in \Expr \Gamma A$ in a PCA~$A$ at a
  valuation $\eta : \Gamma \to A$ is defined recursively by the clauses
  %
  \begin{align*}
    \eval{\combK}{\eta} &= \combK \\
    \eval{\combS}{\eta} &= \combK \\
    \eval{a}{\eta} &= a  &&\text{if $a \in A$}\\
    \eval{x}{\eta} &= \eta(x)  &&\text{if $x \in \Gamma$}\\
    \eval{e_1 \cdot e_2}{\eta} &= (\eval{e_1} \eta) \cdot_A (\eval{e_2} \eta).
  \end{align*}
\end{definition}

\begin{definition}
  \label{def:expression-defined}
  %
  \leanok
  \lean{Expr.defined}
  \uses{def:expression, def:evaluation}
  %
  An expression $e \in \Expr \Gamma A$ is \emph{defined} when $\defined{\eval{e}{\eta}}$
  for all $\eta : \Gamma \to A$.
\end{definition}


\begin{definition}
  \label{def:abstraction}
  %
  \leanok
  \lean{Expr.abstr}
  \uses{def:expression, def:override, def:PCA}
  %
  The \emph{abstraction} of an expression $e \in \Expr \Gamma A$ 
  with respect to a variable $x \in \Gamma$
  is the expression $\abstr{x} e \in \Expr \Gamma A$ defined recursively by
  %
  \begin{align*}
    \abstr{x} x &= \combS \cdot \combK \cdot \combK, \\
    \abstr{x} y &= \combK \cdot y &&\text{if $x \neq y \in \Gamma$}, \\
    \abstr{x} a &= \combK \cdot a &&\text{if $a \in A$}, \\
    \abstr{x} (e_1 \cdot e_2) &= \combS \cdot (\abstr{x} e_1) \cdot (\abstr{x} e_2).
  \end{align*}
\end{definition}


\begin{proposition}
  \label{prop:abstraction-defined}
  %
  \leanok
  \lean{Expr.df_abstr}
  \uses{def:expression-defined, def:abstraction}
  %
  An abstraction $\abstr{x} e$ is defined.
\end{proposition}

\begin{proof}
  \leanok
  By straightforward structural induction on~$e$.
\end{proof}

\begin{proposition}
  \label{prop:abstraction-equal}
  %
  \leanok
  \lean{Expr.eval_abstr}
  \uses{def:expression, def:evaluation, def:abstraction, def:override}
  %
  Let $x \in \Gamma$ and $e \in \Expr {(\Gamma \cup \{x\})} A$.
  Then for every $a \in A$ and $\eta : \Gamma \to A$
  %
  \begin{equation*}
    \eval{(\abstr{x} e) \cdot a}{\eta} =
    \eval{e}{\override{\eta}{x}{a}}.
  \end{equation*}
\end{proposition}

\begin{proof}
  \leanok
  By straightforward structural induction on~$e$.
\end{proof}

\section{Programming with PCAs}

We next show how to write programs in a PCA~$A$.
We call the elements of~$A$ accomplishing various programming tasks \emph{combinators}.

\begin{definition}[Identity]
  \label{def:combinator-I}
  \lean{PCA.I, PCA.eq_I}
  \leanok
  \uses{def:PCA}
  %
  There is the \emph{identity combinator} $\comb{I} \in A$ such that $I \, a = a$ for all $a \in A$.
\end{definition}

\begin{definition}[Pairing]
  \label{def:combinator-pairing}
  \uses{def:PCA}
  \lean{PCA.pair, PCA.fst, PCA.snd, PCA.eq_fst_pair, PCA.eq_snd_pair}
  \leanok
  %
  There are combinators $\comb{pair}, \comb{fst}, \comb{snd} \in A$ such that, for all $a, b \in A$,
  %
  \begin{align*}
    \comb{fst}\, (\comb{pair}\,a\,b) &= a,
    &
    \comb{snd}\, (\comb{pair}\,a\,b) &= b.
  \end{align*}
\end{definition}

\begin{definition}[Booleans]
  \label{def:combinator-bool}
  \lean{PCA.ite, PCA.fal, PCA.tru, PCA.eq_ite_fal, PCA.eq_ite_tru}
  \uses{def:PCA}
  \leanok
  %
  There are combinators $\comb{ite} \, \comb{fal}, \comb{tru} \in A$ such that, for all $a, b \in a$,
  %
  \begin{equation*}
    \comb{ite} \, \comb{tru} \ ,a \, b = a
    \text{and}
    \comb{ite} \, \comb{fal} \, a \, b = b.
  \end{equation*}
\end{definition}

\begin{definition}[Numerals]
  \label{def:combinator-nat}
  \lean{PCA.numeral, PCA.succ, PCA.primrec, PCA.eq_primrec_zero, PCA.eq_primrec_succ}
  \uses{def:PCA}
  \leanok
  %
  For each $n \in \NN$ there is $\overline{n} \in A$, as well as combinators $\comb{succ}, \comb{primrec} \in A$
  such that, for all $n \in \NN$ and $a, f \in A$,
  %
  \begin{align*}
    \comb{succ} \, \numeral{n} &= \numeral{n+1},
    \\
    \comb{primrec} \, a \, f \, \numeral{0} &= a,
    \\
    \comb{primrec} \, a \, f \, \numeral{n+1} &= f \, \numeral{n} \, (\comb{primrec} \, a \, f \, \numeral{n}).
  \end{align*}
\end{definition}

\begin{definition}
  %
  \label{def:combinator-Y}
  \uses{def:PCA}
  \lean{PCA.Y, PCA.df_Y, PCA.eq_Y}
  \leanok
  %
  There is a combinator $\comb{Y} \in A$ such that, for all $a \in A$,
  %
  \begin{equation*}
    \comb{Y} \, a = a (\comb{Z} \, a).
  \end{equation*}
\end{definition}

\begin{definition}[General recursion]
  %
  \label{def:combinator-Z}
  \uses{def:PCA}
  \lean{PCA.Z, PCA.df_Z, PCA.eq_Z}
  \leanok
  %
  There is a combinator $\comb{Z} \in A$ such that, for all $f, a \in A$,
  %
  \begin{equation*}
    \defined{\comb{Z} \, f},
    \qquad\text{and}\qquad
    \comb{Z} \, f \, a = f (\comb{Z} \, f) \, a.
  \end{equation*}
\end{definition}

\begin{definition}
  \label{def:representable-function}
  \uses{def:PCA}
  %
  A partial map $f : A \parto A$ is \emph{represented} by $a \in A$ when,
  for all $b \in B$, $f(b) = a \cdot b$.
\end{definition}

\begin{theorem}
  \label{thm:recursive-map-representable}
  \uses{def:representable-function}
  Every general recursive map $f : \NN \parto \NN$ is represented in the following sense:
  there is $a \in A$ such that, for all $n \in \NN$, if $\defined{f(n)}$ then $\numeral{f(n)} = a \cdot \numeral{n}$.
\end{theorem}

\section{Examples of PCAs}

\subsection{Untyped $\lambda$-calculus}
\label{sec:untyped-lambda-calculus}

\subsection{Kleene's first algebra}
\label{sec:kleenes-first-algebra}

\subsection{Scott's graph model $\mathcal{P}(\omega)$}
\label{sec:scotts-graph-model}



%%% Local Variables:
%%% mode: LaTeX
%%% TeX-master: "print"
%%% End:
